\documentclass[a4paper,10pt]{article}
\usepackage[utf8]{inputenc}      % accents
\usepackage[T1]{fontenc}
\usepackage[pdftex]{graphicx}
\usepackage{subfigure}       % graphiques
\usepackage{amsmath}               % texte dans les equations
\usepackage{indentfirst}           % retrait de paragraphes
\usepackage[english]{babel}         % langue
%\usepackage{vmargin}               % mise en page
\usepackage{verbatim}              % inclusion de codes
\usepackage{makeidx} 
\usepackage[]{algorithm2e}
%\usepackage{supertabular}          % gros tableaux
%\usepackage{rotating}

%\setpapersize{A4}
\date{GP3 - 2018}

\newlength\epaisLigne
\newcommand\Ghline{\noalign{\global\epaisLigne\arrayrulewidth\global\arrayrulewidth 1pt} \hline \noalign{\global\arrayrulewidth\epaisLigne}}

\SetKwIF{If}{ElseIf}{Else}{if}{then}{elseif}{else}{end}
\SetKwFor{While}{while}{}{end}

\SetKwFor{For}{for}{}{end}
%\makeindex
\textwidth 15 cm \oddsidemargin 0 cm
\begin{document}
\title {Solving the 1D Heat equation\\Modelling the heat evolution inside a wall}
\maketitle
%\begin{minipage}{\textwidth}
%\printindex
%\end{minipage}

The Heat equation is a Partial Differential Equation (PDE) modelling the evolution of the heat inside a solid body:

$$\frac{\partial T}{\partial t}=\frac{1}{\rho C}\nabla.\Big(k\nabla T\Big) $$

With the thermal conductivity $k$, the density $\rho$ and the specific heat capacity $C$.\\
In the 1D case and with $k$ constant, it can be reduced to:

$$\frac{\partial T}{\partial t}=\frac{k}{\rho C}\frac{\partial^2 T}{\partial x^2}$$

We will study a wall of width $e=35cm$ of concrete ($k=0.28W/m/K$, $\rho=600kg.m^{-3}$ and $C=1kJ/K/kg$) with two different domains (exterior on the left and interior on the right) of respective temperature $T_{ext}$ and $T_{int}$ on each side :
\begin{itemize}
\item $T_{int} = 300 K$
\item $T_{ext} =  290 - 5 \sin(\frac{2 \pi t}{24 \times 3600})$\\
\end{itemize}

In order to keep thing simple for today, we will use finite difference, and more precisely:
\begin{itemize}
\item Forward first order finite difference in time (also called explicit Euler )
\item Central second order finite difference in space
\item First order finite difference in space at the boundaries\\
\end{itemize}

In time we will need an initial condition, wich is chosen to be constant in all the wall : $T(x,t=0)=T_{ext}$.\\
Moreover, the time step $\Delta t$ is constrained. We will take $\Delta t=\frac{\Delta x^2}{4a}$ with $a=\frac{k}{\rho C}$\\

In space we will need two boundary conditions. In order to impose them, the first point and the last point will answer to a specific equation, one of: 
\begin{itemize}
\item Dirichlet: $T = T_{\text{bnd}}$
\item Neuman: $-k\nabla T =\phi_{\text{bnd}}$
\item Robin: $-k\nabla T + hT=\phi_{\text{bnd}} + h T_{\text{bnd}}$
\end{itemize}
with a solar flux $\phi_{ext}=10W.m^{-2}$, an interior flux $\phi_{int}=100W.m^{-2}$ and a convective coefficient $h=10W.m^{-2}.K^{-1}$.\\

\newpage
The objective of your work is to test the following sets of boundary conditions:
\begin{enumerate}
\item Dirichlet on both sides
\item Neumann on the left and Dirichlet on the right
\item Robin on the left and Dirichlet on the right
\item Robin on both sides\\
\end{enumerate}

The complete program should be similar to that:\\
\noindent\fbox{\parbox{\linewidth \fboxrule-2\fboxsep}{
\begin{algorithm}[H]
 %\KwData{this text}
 
Initialize physical constants \;
Initialize parameters (tmax, precision, $nx$, etc.)  \;
Initialize numerical constants ($\Delta x$, $\Delta t$, etc.))\;
Initialize variables: t, T, etc.\;
\While{$t<tmax$}{
 Compute $T_{ext}$ \;
 Compute $T$ inside (except boundary conditions) \;
 Compute $T$ at the boundaries \;
 Advance time \;
}
Print valuable infos \;
Plot the evolution of the temperature norm in time along with the exterior temperature\;
Plot the final temperature in space\;
\end{algorithm}
}}\\~\\

The temperature norm  will be computed as $\left(\frac{1}{e}\int_{0}^{e} T^2dx\right)^\frac{1}{2}$\\

Complete the following tasks (in any order you want)
\begin{itemize}
\item 	Write a file tools.py containing:
\begin{itemize}
\item 	a function norm($T$, $\Delta x$, $e$) that return the norm of the current temperature
\end{itemize}
\item 	Write a file finite.py containing:
\begin{itemize}
\item 	a function dder($T$, $\Delta x$) that return the second derivative of T inside
\item 	the functions apply\_bnd\_i($T$, $T_{int}$, $T_{ext}$, $k$, $h$, $\phi_{int}$, $\phi_{ext}$, $\Delta x$)\\
that apply the ith boundary condition to $T$ and return nothing
\end{itemize}
\item 	Write a file equation.py containing:
\begin{itemize}
\item 	a function heat($T$, $\Delta t$, $\Delta x$, $k$, $\rho$, $C$) that compute a new T inside (and zeros at the boundaries) and return this new T
\end{itemize}
\item 	Write a file program.py containing:
\begin{itemize}
\item       a function T\_ext(t) that compute the exterior temperature
\item 	the main program \\
\end{itemize}
\end{itemize}
\newpage

Bellow your program add a large comment, answering the following questions:
\begin{enumerate}
\item How much time does it take to converge toward a ``stationnary'' state ?
\item Does the temperature of the wall and the exterior temperature have the same frequency ? Why ?
\item Does the temperature of the wall and the exterior temperature are synchronized ? Why ?\\
\item Does the number of points have an influence on the final temperature ?
\item What is the order of convergence of our discretisation ?
\item Why the time step is constrained ?\\
\item Can you give a physical interpretation of the 3 type of boundary conditions\\~\\
\end{enumerate}
%
%The notation may (or may not) be as follow
%\begin{itemize}
%\item 1pt if the code \textbf{runs} (whatever might be the results)
%\item 1pt if the code gives the good results (on one case)
%\item 1pt for each good plot of each set of boundary conditions (8pt in total)
%\item 3pt for each function (good result, well written, commented) (24pt in total)
%\item 5pt for the main program (good result, well written, commented)
%\item 1pt for questions 1,2 and 3
%\item 2pt for questions 4, 5 and 6
%\item 3pt for questions 7
%\end{itemize}
%For a total of 51pt
\end{document}
